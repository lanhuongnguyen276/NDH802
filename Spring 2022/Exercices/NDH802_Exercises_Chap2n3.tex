% Options for packages loaded elsewhere
\PassOptionsToPackage{unicode}{hyperref}
\PassOptionsToPackage{hyphens}{url}
%
\documentclass[
]{article}
\usepackage{amsmath,amssymb}
\usepackage{lmodern}
\usepackage{ifxetex,ifluatex}
\ifnum 0\ifxetex 1\fi\ifluatex 1\fi=0 % if pdftex
  \usepackage[T1]{fontenc}
  \usepackage[utf8]{inputenc}
  \usepackage{textcomp} % provide euro and other symbols
\else % if luatex or xetex
  \usepackage{unicode-math}
  \defaultfontfeatures{Scale=MatchLowercase}
  \defaultfontfeatures[\rmfamily]{Ligatures=TeX,Scale=1}
\fi
% Use upquote if available, for straight quotes in verbatim environments
\IfFileExists{upquote.sty}{\usepackage{upquote}}{}
\IfFileExists{microtype.sty}{% use microtype if available
  \usepackage[]{microtype}
  \UseMicrotypeSet[protrusion]{basicmath} % disable protrusion for tt fonts
}{}
\makeatletter
\@ifundefined{KOMAClassName}{% if non-KOMA class
  \IfFileExists{parskip.sty}{%
    \usepackage{parskip}
  }{% else
    \setlength{\parindent}{0pt}
    \setlength{\parskip}{6pt plus 2pt minus 1pt}}
}{% if KOMA class
  \KOMAoptions{parskip=half}}
\makeatother
\usepackage{xcolor}
\IfFileExists{xurl.sty}{\usepackage{xurl}}{} % add URL line breaks if available
\IfFileExists{bookmark.sty}{\usepackage{bookmark}}{\usepackage{hyperref}}
\hypersetup{
  pdftitle={NDH802 Solutions to rec. exercises Chap 2 and 3},
  pdfauthor={Huong},
  hidelinks,
  pdfcreator={LaTeX via pandoc}}
\urlstyle{same} % disable monospaced font for URLs
\usepackage[margin=1in]{geometry}
\usepackage{color}
\usepackage{fancyvrb}
\newcommand{\VerbBar}{|}
\newcommand{\VERB}{\Verb[commandchars=\\\{\}]}
\DefineVerbatimEnvironment{Highlighting}{Verbatim}{commandchars=\\\{\}}
% Add ',fontsize=\small' for more characters per line
\usepackage{framed}
\definecolor{shadecolor}{RGB}{248,248,248}
\newenvironment{Shaded}{\begin{snugshade}}{\end{snugshade}}
\newcommand{\AlertTok}[1]{\textcolor[rgb]{0.94,0.16,0.16}{#1}}
\newcommand{\AnnotationTok}[1]{\textcolor[rgb]{0.56,0.35,0.01}{\textbf{\textit{#1}}}}
\newcommand{\AttributeTok}[1]{\textcolor[rgb]{0.77,0.63,0.00}{#1}}
\newcommand{\BaseNTok}[1]{\textcolor[rgb]{0.00,0.00,0.81}{#1}}
\newcommand{\BuiltInTok}[1]{#1}
\newcommand{\CharTok}[1]{\textcolor[rgb]{0.31,0.60,0.02}{#1}}
\newcommand{\CommentTok}[1]{\textcolor[rgb]{0.56,0.35,0.01}{\textit{#1}}}
\newcommand{\CommentVarTok}[1]{\textcolor[rgb]{0.56,0.35,0.01}{\textbf{\textit{#1}}}}
\newcommand{\ConstantTok}[1]{\textcolor[rgb]{0.00,0.00,0.00}{#1}}
\newcommand{\ControlFlowTok}[1]{\textcolor[rgb]{0.13,0.29,0.53}{\textbf{#1}}}
\newcommand{\DataTypeTok}[1]{\textcolor[rgb]{0.13,0.29,0.53}{#1}}
\newcommand{\DecValTok}[1]{\textcolor[rgb]{0.00,0.00,0.81}{#1}}
\newcommand{\DocumentationTok}[1]{\textcolor[rgb]{0.56,0.35,0.01}{\textbf{\textit{#1}}}}
\newcommand{\ErrorTok}[1]{\textcolor[rgb]{0.64,0.00,0.00}{\textbf{#1}}}
\newcommand{\ExtensionTok}[1]{#1}
\newcommand{\FloatTok}[1]{\textcolor[rgb]{0.00,0.00,0.81}{#1}}
\newcommand{\FunctionTok}[1]{\textcolor[rgb]{0.00,0.00,0.00}{#1}}
\newcommand{\ImportTok}[1]{#1}
\newcommand{\InformationTok}[1]{\textcolor[rgb]{0.56,0.35,0.01}{\textbf{\textit{#1}}}}
\newcommand{\KeywordTok}[1]{\textcolor[rgb]{0.13,0.29,0.53}{\textbf{#1}}}
\newcommand{\NormalTok}[1]{#1}
\newcommand{\OperatorTok}[1]{\textcolor[rgb]{0.81,0.36,0.00}{\textbf{#1}}}
\newcommand{\OtherTok}[1]{\textcolor[rgb]{0.56,0.35,0.01}{#1}}
\newcommand{\PreprocessorTok}[1]{\textcolor[rgb]{0.56,0.35,0.01}{\textit{#1}}}
\newcommand{\RegionMarkerTok}[1]{#1}
\newcommand{\SpecialCharTok}[1]{\textcolor[rgb]{0.00,0.00,0.00}{#1}}
\newcommand{\SpecialStringTok}[1]{\textcolor[rgb]{0.31,0.60,0.02}{#1}}
\newcommand{\StringTok}[1]{\textcolor[rgb]{0.31,0.60,0.02}{#1}}
\newcommand{\VariableTok}[1]{\textcolor[rgb]{0.00,0.00,0.00}{#1}}
\newcommand{\VerbatimStringTok}[1]{\textcolor[rgb]{0.31,0.60,0.02}{#1}}
\newcommand{\WarningTok}[1]{\textcolor[rgb]{0.56,0.35,0.01}{\textbf{\textit{#1}}}}
\usepackage{longtable,booktabs,array}
\usepackage{calc} % for calculating minipage widths
% Correct order of tables after \paragraph or \subparagraph
\usepackage{etoolbox}
\makeatletter
\patchcmd\longtable{\par}{\if@noskipsec\mbox{}\fi\par}{}{}
\makeatother
% Allow footnotes in longtable head/foot
\IfFileExists{footnotehyper.sty}{\usepackage{footnotehyper}}{\usepackage{footnote}}
\makesavenoteenv{longtable}
\usepackage{graphicx}
\makeatletter
\def\maxwidth{\ifdim\Gin@nat@width>\linewidth\linewidth\else\Gin@nat@width\fi}
\def\maxheight{\ifdim\Gin@nat@height>\textheight\textheight\else\Gin@nat@height\fi}
\makeatother
% Scale images if necessary, so that they will not overflow the page
% margins by default, and it is still possible to overwrite the defaults
% using explicit options in \includegraphics[width, height, ...]{}
\setkeys{Gin}{width=\maxwidth,height=\maxheight,keepaspectratio}
% Set default figure placement to htbp
\makeatletter
\def\fps@figure{htbp}
\makeatother
\setlength{\emergencystretch}{3em} % prevent overfull lines
\providecommand{\tightlist}{%
  \setlength{\itemsep}{0pt}\setlength{\parskip}{0pt}}
\setcounter{secnumdepth}{-\maxdimen} % remove section numbering
\ifluatex
  \usepackage{selnolig}  % disable illegal ligatures
\fi

\title{NDH802 Solutions to rec. exercises Chap 2 and 3}
\author{Huong}
\date{}

\begin{document}
\maketitle

\hypertarget{section}{%
\subsection{1.4}\label{section}}

\begin{Shaded}
\begin{Highlighting}[]
\CommentTok{\#Data}
\NormalTok{price }\OtherTok{=} \FunctionTok{c}\NormalTok{(}\DecValTok{104}\NormalTok{, }\DecValTok{135}\NormalTok{, }\DecValTok{80}\NormalTok{, }\DecValTok{200}\NormalTok{, }\DecValTok{98}\NormalTok{, }\DecValTok{206}\NormalTok{, }\DecValTok{141}\NormalTok{, }\DecValTok{109}\NormalTok{)}
\NormalTok{no.apt }\OtherTok{=} \FunctionTok{c}\NormalTok{(}\DecValTok{220}\NormalTok{, }\DecValTok{380}\NormalTok{,}\DecValTok{350}\NormalTok{, }\DecValTok{100}\NormalTok{, }\DecValTok{440}\NormalTok{, }\DecValTok{185}\NormalTok{, }\DecValTok{250}\NormalTok{, }\DecValTok{120}\NormalTok{)}
\FunctionTok{plot}\NormalTok{(price, no.apt)}
\end{Highlighting}
\end{Shaded}

\includegraphics{NDH802_Exercises_Chap2n3_files/figure-latex/unnamed-chunk-1-1.pdf}

\hypertarget{section-1}{%
\subsection{2.4}\label{section-1}}

\begin{Shaded}
\begin{Highlighting}[]
\NormalTok{data\_q2}\FloatTok{.4} \OtherTok{=} \FunctionTok{c}\NormalTok{(}\FloatTok{2.51}\NormalTok{, }\FloatTok{3.74}\NormalTok{, }\FloatTok{4.15}\NormalTok{, }\FloatTok{5.33}\NormalTok{, }\FloatTok{6.18}\NormalTok{, }\FloatTok{6.65}\NormalTok{, }\FloatTok{6.92}\NormalTok{, }\FloatTok{6.95}\NormalTok{, }\FloatTok{7.18}\NormalTok{, }\FloatTok{7.54}\NormalTok{)}
\FunctionTok{mean}\NormalTok{(data\_q2}\FloatTok{.4}\NormalTok{)}
\end{Highlighting}
\end{Shaded}

\begin{verbatim}
## [1] 5.715
\end{verbatim}

\begin{Shaded}
\begin{Highlighting}[]
\FunctionTok{median}\NormalTok{(data\_q2}\FloatTok{.4}\NormalTok{)}
\end{Highlighting}
\end{Shaded}

\begin{verbatim}
## [1] 6.415
\end{verbatim}

\hypertarget{section-2}{%
\subsection{2.12}\label{section-2}}

\begin{Shaded}
\begin{Highlighting}[]
\NormalTok{data\_q2}\FloatTok{.12} \OtherTok{=} \FunctionTok{c}\NormalTok{(}\DecValTok{5}\NormalTok{, }\DecValTok{9}\NormalTok{, }\DecValTok{10}\NormalTok{, }\DecValTok{2}\NormalTok{, }\DecValTok{7}\NormalTok{, }\DecValTok{9}\NormalTok{, }\DecValTok{14}\NormalTok{)}
\FunctionTok{var}\NormalTok{(data\_q2}\FloatTok{.12}\NormalTok{)}
\end{Highlighting}
\end{Shaded}

\begin{verbatim}
## [1] 14.66667
\end{verbatim}

\begin{Shaded}
\begin{Highlighting}[]
\FunctionTok{sd}\NormalTok{(data\_q2}\FloatTok{.12}\NormalTok{)}
\end{Highlighting}
\end{Shaded}

\begin{verbatim}
## [1] 3.829708
\end{verbatim}

\hypertarget{section-3}{%
\subsection{2.20}\label{section-3}}

\begin{Shaded}
\begin{Highlighting}[]
\NormalTok{eur\_usd }\OtherTok{=} \FunctionTok{c}\NormalTok{(}\FloatTok{1.1410}\NormalTok{, }\FloatTok{1.1363}\NormalTok{, }\FloatTok{1.1351}\NormalTok{, }\FloatTok{1.1324}\NormalTok{, }\FloatTok{1.1276}\NormalTok{, }\FloatTok{1.1332}\NormalTok{, }\FloatTok{1.1266}\NormalTok{)}
\NormalTok{usd\_jyp }\OtherTok{=} \FunctionTok{c}\NormalTok{(}\FloatTok{109.95}\NormalTok{, }\FloatTok{109.96}\NormalTok{, }\FloatTok{109.80}\NormalTok{, }\FloatTok{109.77}\NormalTok{, }\FloatTok{110.41}\NormalTok{, }\FloatTok{110.48}\NormalTok{, }\FloatTok{110.00}\NormalTok{)}
\FunctionTok{mean}\NormalTok{(eur\_usd) }\SpecialCharTok{\textless{}} \FunctionTok{mean}\NormalTok{(usd\_jyp)}
\end{Highlighting}
\end{Shaded}

\begin{verbatim}
## [1] TRUE
\end{verbatim}

\begin{Shaded}
\begin{Highlighting}[]
\FunctionTok{sd}\NormalTok{(eur\_usd) }\SpecialCharTok{\textless{}} \FunctionTok{sd}\NormalTok{(usd\_jyp)}
\end{Highlighting}
\end{Shaded}

\begin{verbatim}
## [1] TRUE
\end{verbatim}

\hypertarget{section-4}{%
\subsection{3.7}\label{section-4}}

a.

\begin{longtable}[]{@{}lllll@{}}
\toprule
Pairs & & & & \\
\midrule
\endhead
1. M1,M2 & 5. M2,M1 & 9. M3,M1 & 13. T1,M1 & 17. T2,M1 \\
2. M1,M3 & 6. M2,M3 & 10.M3,M2 & 14. T1,M2 & 18. T2,M2 \\
3. M1,T1 & 7.~ M2,T1 & 11. M3,T1 & 15. T1,M3 & 19. T2,M3 \\
4. M1,T2 & 8.~ M2,T2 & 12. M3,T2 & 16. T1,T2 & 20. T2,T1 \\
\bottomrule
\end{longtable}

b. Event A is that at least one of the two cars selected is a Toyota.
Outcomes 3, 4, 7, 8, 11-20.

c. Event B is that the two cars selected are of the same model. Outcomes
1, 2, 5, 6, 9, 10, 16, 20.

d. The complement of A is the event that the customers do not select at
least one Toyota. That is, no one chose Toyota. Equivalently, both of
them chose Mercedes. Outcomes 1, 2, 5, 6, 9, 10.

e. \((A \cap B) \cup (\bar{A} \cap B) = (A \cup \bar{A}) \cap B = B\).
Alternatively, we can see:

\begin{itemize}
\item
  \((A \cap B)\): Outcomes 16, 20.
\item
  \((\bar{A} \cap B)\): Outcomes 1, 2, 5, 6, 9, 10.
\item
  \((A \cap B) \cup (\bar{A} \cap B)\): Outcomes 1, 2, 5, 6, 9, 10, 16,
  or 20; which is event B.
\end{itemize}

d.
\(A \cup (\bar{A} \cap B) = (A \cup \bar{A}) \cap (A \cup B) = A \cup B\).
Alternatively, we can see:

\begin{itemize}
\item
  \((\bar{A} \cap B)\): Outcomes 1, 2, 5, 6, 9, 10.
\item
  \(A \cup (\bar{A} \cap B)\): Outcomes 1-20, which is \(A \cup B\).
\end{itemize}

\hypertarget{section-5}{%
\subsection{3.19}\label{section-5}}

\[\begin{align*}
P( A \cap B) &= P(A) + P(B) - P(A \cup B) \\ &= 0.3 + 0.7 - 0.9 = 0.1
\end{align*}\]

\hypertarget{section-6}{%
\subsection{3.98}\label{section-6}}

Assumptions: \(P(MBA) = 0.35\), \(P(\>35 yo+) = 0.4\),
\(P(35yo+ \mid MBA) = 0.3\)

a. \(P(MBA \cap 35yo+)\).

b. \(P(MBA \mid 35yo+)\).

c. \(P(MBA \cup 35yo+)\).

d. \(P(\bar{MBA} \mid 35yo+)\).

e. Are the events MBA and over age 35 independent?

f. Are the events MBA and over age 35 mutually exclusive?

g. Are the events MBA and over age 35 collectively exhaustive?

\hypertarget{section-7}{%
\subsection{3.108}\label{section-7}}

Let A be the event where the passengers are carrying more liquor than is
allowed and B be the event where the TPS identifies it. We have
\(P(A) = 0.2, P(B \mid A) = 0.8\) and \(P(B|\bar{A}) = 0.2\). By Bayes's
theorem:

\[\begin{align*}
P(A \mid B)
& = \frac{P(B \mid A) * P(A)}{P(B)} \\
& = \frac{P(B \mid A)*P(A)}{P(B \mid A)*P(A) + P(B \mid \bar{A})*P(\bar{A})} \\
& = \frac{0.8*0.2}{0.8*0.2 + 0.2*0.8} = 0.5
\end{align*}\] In case you wonder, this is normally called the law of
total probability, which is quite handy:

\[\begin{align*}
P(B) & = P(A \cap B) + P(\bar{A} \cap B) \quad \text{(See Exercise 3.7e)}\\
& = P(B \mid A)*P(A) + P(B \mid \bar{A})*P(\bar{A})
\end{align*}\]

\end{document}
