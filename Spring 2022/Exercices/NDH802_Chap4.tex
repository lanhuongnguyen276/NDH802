% Options for packages loaded elsewhere
\PassOptionsToPackage{unicode}{hyperref}
\PassOptionsToPackage{hyphens}{url}
%
\documentclass[
]{article}
\usepackage{amsmath,amssymb}
\usepackage{lmodern}
\usepackage{ifxetex,ifluatex}
\ifnum 0\ifxetex 1\fi\ifluatex 1\fi=0 % if pdftex
  \usepackage[T1]{fontenc}
  \usepackage[utf8]{inputenc}
  \usepackage{textcomp} % provide euro and other symbols
\else % if luatex or xetex
  \usepackage{unicode-math}
  \defaultfontfeatures{Scale=MatchLowercase}
  \defaultfontfeatures[\rmfamily]{Ligatures=TeX,Scale=1}
\fi
% Use upquote if available, for straight quotes in verbatim environments
\IfFileExists{upquote.sty}{\usepackage{upquote}}{}
\IfFileExists{microtype.sty}{% use microtype if available
  \usepackage[]{microtype}
  \UseMicrotypeSet[protrusion]{basicmath} % disable protrusion for tt fonts
}{}
\makeatletter
\@ifundefined{KOMAClassName}{% if non-KOMA class
  \IfFileExists{parskip.sty}{%
    \usepackage{parskip}
  }{% else
    \setlength{\parindent}{0pt}
    \setlength{\parskip}{6pt plus 2pt minus 1pt}}
}{% if KOMA class
  \KOMAoptions{parskip=half}}
\makeatother
\usepackage{xcolor}
\IfFileExists{xurl.sty}{\usepackage{xurl}}{} % add URL line breaks if available
\IfFileExists{bookmark.sty}{\usepackage{bookmark}}{\usepackage{hyperref}}
\hypersetup{
  pdftitle={NDH802 Solutions to rec. exercises Chap 4 and 5},
  pdfauthor={Huong},
  hidelinks,
  pdfcreator={LaTeX via pandoc}}
\urlstyle{same} % disable monospaced font for URLs
\usepackage[margin=1in]{geometry}
\usepackage{color}
\usepackage{fancyvrb}
\newcommand{\VerbBar}{|}
\newcommand{\VERB}{\Verb[commandchars=\\\{\}]}
\DefineVerbatimEnvironment{Highlighting}{Verbatim}{commandchars=\\\{\}}
% Add ',fontsize=\small' for more characters per line
\usepackage{framed}
\definecolor{shadecolor}{RGB}{248,248,248}
\newenvironment{Shaded}{\begin{snugshade}}{\end{snugshade}}
\newcommand{\AlertTok}[1]{\textcolor[rgb]{0.94,0.16,0.16}{#1}}
\newcommand{\AnnotationTok}[1]{\textcolor[rgb]{0.56,0.35,0.01}{\textbf{\textit{#1}}}}
\newcommand{\AttributeTok}[1]{\textcolor[rgb]{0.77,0.63,0.00}{#1}}
\newcommand{\BaseNTok}[1]{\textcolor[rgb]{0.00,0.00,0.81}{#1}}
\newcommand{\BuiltInTok}[1]{#1}
\newcommand{\CharTok}[1]{\textcolor[rgb]{0.31,0.60,0.02}{#1}}
\newcommand{\CommentTok}[1]{\textcolor[rgb]{0.56,0.35,0.01}{\textit{#1}}}
\newcommand{\CommentVarTok}[1]{\textcolor[rgb]{0.56,0.35,0.01}{\textbf{\textit{#1}}}}
\newcommand{\ConstantTok}[1]{\textcolor[rgb]{0.00,0.00,0.00}{#1}}
\newcommand{\ControlFlowTok}[1]{\textcolor[rgb]{0.13,0.29,0.53}{\textbf{#1}}}
\newcommand{\DataTypeTok}[1]{\textcolor[rgb]{0.13,0.29,0.53}{#1}}
\newcommand{\DecValTok}[1]{\textcolor[rgb]{0.00,0.00,0.81}{#1}}
\newcommand{\DocumentationTok}[1]{\textcolor[rgb]{0.56,0.35,0.01}{\textbf{\textit{#1}}}}
\newcommand{\ErrorTok}[1]{\textcolor[rgb]{0.64,0.00,0.00}{\textbf{#1}}}
\newcommand{\ExtensionTok}[1]{#1}
\newcommand{\FloatTok}[1]{\textcolor[rgb]{0.00,0.00,0.81}{#1}}
\newcommand{\FunctionTok}[1]{\textcolor[rgb]{0.00,0.00,0.00}{#1}}
\newcommand{\ImportTok}[1]{#1}
\newcommand{\InformationTok}[1]{\textcolor[rgb]{0.56,0.35,0.01}{\textbf{\textit{#1}}}}
\newcommand{\KeywordTok}[1]{\textcolor[rgb]{0.13,0.29,0.53}{\textbf{#1}}}
\newcommand{\NormalTok}[1]{#1}
\newcommand{\OperatorTok}[1]{\textcolor[rgb]{0.81,0.36,0.00}{\textbf{#1}}}
\newcommand{\OtherTok}[1]{\textcolor[rgb]{0.56,0.35,0.01}{#1}}
\newcommand{\PreprocessorTok}[1]{\textcolor[rgb]{0.56,0.35,0.01}{\textit{#1}}}
\newcommand{\RegionMarkerTok}[1]{#1}
\newcommand{\SpecialCharTok}[1]{\textcolor[rgb]{0.00,0.00,0.00}{#1}}
\newcommand{\SpecialStringTok}[1]{\textcolor[rgb]{0.31,0.60,0.02}{#1}}
\newcommand{\StringTok}[1]{\textcolor[rgb]{0.31,0.60,0.02}{#1}}
\newcommand{\VariableTok}[1]{\textcolor[rgb]{0.00,0.00,0.00}{#1}}
\newcommand{\VerbatimStringTok}[1]{\textcolor[rgb]{0.31,0.60,0.02}{#1}}
\newcommand{\WarningTok}[1]{\textcolor[rgb]{0.56,0.35,0.01}{\textbf{\textit{#1}}}}
\usepackage{longtable,booktabs,array}
\usepackage{calc} % for calculating minipage widths
% Correct order of tables after \paragraph or \subparagraph
\usepackage{etoolbox}
\makeatletter
\patchcmd\longtable{\par}{\if@noskipsec\mbox{}\fi\par}{}{}
\makeatother
% Allow footnotes in longtable head/foot
\IfFileExists{footnotehyper.sty}{\usepackage{footnotehyper}}{\usepackage{footnote}}
\makesavenoteenv{longtable}
\usepackage{graphicx}
\makeatletter
\def\maxwidth{\ifdim\Gin@nat@width>\linewidth\linewidth\else\Gin@nat@width\fi}
\def\maxheight{\ifdim\Gin@nat@height>\textheight\textheight\else\Gin@nat@height\fi}
\makeatother
% Scale images if necessary, so that they will not overflow the page
% margins by default, and it is still possible to overwrite the defaults
% using explicit options in \includegraphics[width, height, ...]{}
\setkeys{Gin}{width=\maxwidth,height=\maxheight,keepaspectratio}
% Set default figure placement to htbp
\makeatletter
\def\fps@figure{htbp}
\makeatother
\setlength{\emergencystretch}{3em} % prevent overfull lines
\providecommand{\tightlist}{%
  \setlength{\itemsep}{0pt}\setlength{\parskip}{0pt}}
\setcounter{secnumdepth}{-\maxdimen} % remove section numbering
\ifluatex
  \usepackage{selnolig}  % disable illegal ligatures
\fi

\title{NDH802 Solutions to rec. exercises Chap 4 and 5}
\author{Huong}
\date{}

\begin{document}
\maketitle

\hypertarget{section}{%
\subsection{4.13}\label{section}}

The number of computers sold per day at Dan's Computer Works is defined
by the following probability distribution:

\begin{longtable}[]{@{}
  >{\raggedright\arraybackslash}p{(\columnwidth - 14\tabcolsep) * \real{0.12}}
  >{\raggedright\arraybackslash}p{(\columnwidth - 14\tabcolsep) * \real{0.10}}
  >{\raggedright\arraybackslash}p{(\columnwidth - 14\tabcolsep) * \real{0.12}}
  >{\raggedright\arraybackslash}p{(\columnwidth - 14\tabcolsep) * \real{0.12}}
  >{\raggedright\arraybackslash}p{(\columnwidth - 14\tabcolsep) * \real{0.12}}
  >{\raggedright\arraybackslash}p{(\columnwidth - 14\tabcolsep) * \real{0.12}}
  >{\raggedright\arraybackslash}p{(\columnwidth - 14\tabcolsep) * \real{0.12}}
  >{\raggedright\arraybackslash}p{(\columnwidth - 14\tabcolsep) * \real{0.12}}@{}}
\toprule
x & 0 & 1 & 2 & 3 & 4 & 5 & 6 \\
\midrule
\endhead
P(x) & 0.03 & 0.11 & 0.15 & 0.22 & 0.19 & 0.26 & 0.04 \\
\bottomrule
\end{longtable}

a. \(P(3 \le X < 6) = P(X=3) + P(X=4) + P(X=5) = 0.22 + 0.19 + 0.26\)

b.
\(P(X \ge 3) = P(X=3) + P(X=4) + P(X=5) + P(X=6) = 0.22 + 0.19 + 0.26 + 0.04\)

c. \(P(X \le 4) = 1 - P(X=5) - P(X=6) = 1 - 0.26 - 0.04\)

d.
\(P(2 < X \le 5) = P(3 \le X < 6) = P(X=3) + P(X=4) + P(X=5) = 0.22 + 0.19 + 0.26\)

\hypertarget{section-1}{%
\subsection{4.29}\label{section-1}}

An investor is considering three strategies for a \$1,200 investment.
The probable returns are estimated as follows:

\begin{itemize}
\item
  Strategy 1: A certain profit of \$200. That is, \(E(X_1) = 200\),
  \(Var(X_1) = 0\).
\item
  Strategy 2: A profit of \$12,000 with probability 0.15 and a loss of
  \$1,200 with probability 0.85.

  That is,
\end{itemize}

\[
    \begin{align*}
    E(X_2) &= 12000*0.15 - 1200 * 0.85 = 780 \\
    Var(X_2) &= 12000^2*0.15 + (-1200)^2*0.85 - 780^2 = 22,215,600
    \end{align*}
\]

\begin{itemize}
\tightlist
\item
  Strategy 3: A profit of \$500 with probability 0.50, a profit of \$250
  with probability 0.30 and a loss of \$250 with probability 0.20. That
  is,
\end{itemize}

\[
\begin{align*}
    E(X_3) &= 500*0.5 + 250*0.3 - 250 * 0.2 = 275 \\
    Var(X_3) &= 500^2*0.5 + 250^2*0.3 + (-250)^2*0.2 - 275^2 = 80,625
    \end{align*}
\]

Which strategy has the highest expected profit? Strategy 2.

Explain why you would or would not advise the investor to adopt this
strategy. The ``best'' strategy depends on your preference, if you'd
like a high risk high return choice or safe (low risk low return)
choice.

\hypertarget{section-2}{%
\subsection{4.30}\label{section-2}}

Assume \(X \sim Bern(0.8)\). \(E(X) = p = 0.8\).
\(Var(X) = p(1-p) = 0.8*0.2 = 0.16\).

\hypertarget{section-3}{%
\subsection{4.49}\label{section-3}}

A company receives large shipments of parts from two sources. Seventy
percent of the shipments come from a supplier whose shipments typically
contain 10\% defectives, while the remainder are from a supplier whose
shipments typically contain 20\% defectives. A manager receives a
shipment but does not know the source. A random sample of 20 items from
this shipment is tested, and 1 of the parts is found to be defective.
What is the probability that this shipment came from the more reliable
supplier? (Hint: Use Bayes' theorem)

Summarize the questions:

\(A\): Shipment is from reliable supplier. \(P(A) = 0.7\).\\
\(\bar{A}\): Shipment is from the other supplier.
\(P(\bar{A}) = 0.3\).\\
\(X\): Number of defectives.

Find \(P(A \mid X = 1)\).

Now,\\
\(X \mid A \sim Binom(n=20, p = 0.1)\). Hence,
\(P(X=1\mid A) = 0.27\).\\

\begin{Shaded}
\begin{Highlighting}[]
\FunctionTok{dbinom}\NormalTok{(}\AttributeTok{x =} \DecValTok{1}\NormalTok{, }\AttributeTok{size =} \DecValTok{20}\NormalTok{, }\AttributeTok{p =} \FloatTok{0.1}\NormalTok{)}
\end{Highlighting}
\end{Shaded}

\begin{verbatim}
## [1] 0.2701703
\end{verbatim}

\(X \mid \bar{A} \sim Binom(n=20, p = 0.2)\). Hence,
\(P(X=1\mid \bar{A}) = 0.057\).\\

\begin{Shaded}
\begin{Highlighting}[]
\FunctionTok{dbinom}\NormalTok{(}\AttributeTok{x =} \DecValTok{1}\NormalTok{, }\AttributeTok{size =} \DecValTok{20}\NormalTok{, }\AttributeTok{p =} \FloatTok{0.2}\NormalTok{)}
\end{Highlighting}
\end{Shaded}

\begin{verbatim}
## [1] 0.05764608
\end{verbatim}

\[
\begin{align}
P(A \mid X = 1)
&= \frac{P(X = 1 \mid A) * P(A)}{P(X = 1)} \quad \text{(Bayes' theorem)}\\
&= \frac{P(X = 1 \mid A) * P(A)}{P(X = 1 \mid A) * P(A) + P(X = 1 \mid \bar{A})*P(\bar{A})}\\
&= \frac{0.27 * 0.7}{0.27 * 0.7 + 0.057 * 0.3} = 0.917
\end{align}
\]

\hypertarget{section-4}{%
\subsection{5.21}\label{section-4}}

Let \(X \sim N(\mu = 0.2, \sigma^2 = 0.0025)\).

a. \(P(X > 0.4)\).\\

\begin{Shaded}
\begin{Highlighting}[]
\FunctionTok{pnorm}\NormalTok{(}\AttributeTok{q =} \FloatTok{0.4}\NormalTok{, }\AttributeTok{mean =} \FloatTok{0.2}\NormalTok{, }\AttributeTok{sd =} \FunctionTok{sqrt}\NormalTok{(}\FloatTok{0.0025}\NormalTok{), }\AttributeTok{lower.tail =}\NormalTok{ F)}
\end{Highlighting}
\end{Shaded}

\begin{verbatim}
## [1] 3.167124e-05
\end{verbatim}

b. \(P(0.15< X < 0.28)\).\\

\begin{Shaded}
\begin{Highlighting}[]
\FunctionTok{pnorm}\NormalTok{(}\AttributeTok{q =} \FloatTok{0.28}\NormalTok{, }\AttributeTok{mean =} \FloatTok{0.2}\NormalTok{, }\AttributeTok{sd =} \FunctionTok{sqrt}\NormalTok{(}\FloatTok{0.0025}\NormalTok{)) }\SpecialCharTok{{-}} \FunctionTok{pnorm}\NormalTok{(}\AttributeTok{q =} \FloatTok{0.15}\NormalTok{, }\AttributeTok{mean =} \FloatTok{0.2}\NormalTok{, }\AttributeTok{sd =} \FunctionTok{sqrt}\NormalTok{(}\FloatTok{0.0025}\NormalTok{))}
\end{Highlighting}
\end{Shaded}

\begin{verbatim}
## [1] 0.7865455
\end{verbatim}

c. \(P(X < 0.1)\).

\begin{Shaded}
\begin{Highlighting}[]
\FunctionTok{pnorm}\NormalTok{(}\AttributeTok{q =} \FloatTok{0.1}\NormalTok{, }\AttributeTok{mean =} \FloatTok{0.2}\NormalTok{, }\AttributeTok{sd =} \FunctionTok{sqrt}\NormalTok{(}\FloatTok{0.0025}\NormalTok{))}
\end{Highlighting}
\end{Shaded}

\begin{verbatim}
## [1] 0.02275013
\end{verbatim}

d. Find d such that \(P(X > d) = 0.2\)~

\begin{Shaded}
\begin{Highlighting}[]
\FunctionTok{qnorm}\NormalTok{(}\AttributeTok{p =} \FloatTok{0.2}\NormalTok{, }\AttributeTok{mean =} \FloatTok{0.2}\NormalTok{, }\AttributeTok{sd =} \FunctionTok{sqrt}\NormalTok{(}\FloatTok{0.0025}\NormalTok{), }\AttributeTok{lower.tail =}\NormalTok{ F)}
\end{Highlighting}
\end{Shaded}

\begin{verbatim}
## [1] 0.2420811
\end{verbatim}

e. Find e and f such that \(P(e< X < f) = 0.05\) and \(X\) is in the
symmetric interval around the mean

\begin{Shaded}
\begin{Highlighting}[]
\FunctionTok{qnorm}\NormalTok{(}\AttributeTok{p =} \FloatTok{0.475}\NormalTok{, }\AttributeTok{mean =} \FloatTok{0.2}\NormalTok{, }\AttributeTok{sd =} \FunctionTok{sqrt}\NormalTok{(}\FloatTok{0.0025}\NormalTok{))}
\end{Highlighting}
\end{Shaded}

\begin{verbatim}
## [1] 0.1968647
\end{verbatim}

\begin{Shaded}
\begin{Highlighting}[]
\FunctionTok{qnorm}\NormalTok{(}\AttributeTok{p =} \FloatTok{0.475}\NormalTok{, }\AttributeTok{mean =} \FloatTok{0.2}\NormalTok{, }\AttributeTok{sd =} \FunctionTok{sqrt}\NormalTok{(}\FloatTok{0.0025}\NormalTok{), }\AttributeTok{lower.tail =}\NormalTok{ F)}
\end{Highlighting}
\end{Shaded}

\begin{verbatim}
## [1] 0.2031353
\end{verbatim}

\hypertarget{section-5}{%
\subsection{5.87}\label{section-5}}

Let \(X\) be the delivery times in minutes.
\(X \sim N(\mu = 30, \sigma = 6)\).

a. \(P(25 < X < 35)\)\\

\begin{Shaded}
\begin{Highlighting}[]
\FunctionTok{pnorm}\NormalTok{(}\AttributeTok{q =} \DecValTok{35}\NormalTok{, }\AttributeTok{mean =} \DecValTok{30}\NormalTok{, }\AttributeTok{sd =} \DecValTok{6}\NormalTok{) }\SpecialCharTok{{-}} \FunctionTok{pnorm}\NormalTok{(}\AttributeTok{q =} \DecValTok{25}\NormalTok{, }\AttributeTok{mean =} \DecValTok{30}\NormalTok{, }\AttributeTok{sd =} \DecValTok{6}\NormalTok{)}
\end{Highlighting}
\end{Shaded}

\begin{verbatim}
## [1] 0.5953432
\end{verbatim}

b. \(P(X > 45)\) Free pizza

\begin{Shaded}
\begin{Highlighting}[]
\FunctionTok{pnorm}\NormalTok{(}\AttributeTok{q =} \DecValTok{45}\NormalTok{, }\AttributeTok{mean =} \DecValTok{30}\NormalTok{, }\AttributeTok{sd =} \DecValTok{6}\NormalTok{, }\AttributeTok{lower.tail =}\NormalTok{ F)}
\end{Highlighting}
\end{Shaded}

\begin{verbatim}
## [1] 0.006209665
\end{verbatim}

c. Suppose one orders 6 pizzas, find probability that s/he gets at least
one free pizza.\\
We can think of \(Y\) as an event where one gets free pizza. Then
\(Y \sim Binom(n=6, p=0.006)\).

\[
P(Y \ge 1) = 1 - P(Y=0) 
\]

\begin{verbatim}
## [1] 0.0354643
\end{verbatim}

d. Find the shortest interval \([a, b]\) such that
\(P(a < X < b) = 0.8\).

\begin{Shaded}
\begin{Highlighting}[]
\NormalTok{a }\OtherTok{=} \FunctionTok{qnorm}\NormalTok{(}\AttributeTok{p =} \FloatTok{0.1}\NormalTok{, }\AttributeTok{mean =} \DecValTok{30}\NormalTok{, }\AttributeTok{sd =} \DecValTok{6}\NormalTok{)}
\NormalTok{b }\OtherTok{=} \FunctionTok{qnorm}\NormalTok{(}\AttributeTok{p =} \FloatTok{0.1}\NormalTok{, }\AttributeTok{mean =} \DecValTok{30}\NormalTok{, }\AttributeTok{sd =} \DecValTok{6}\NormalTok{, }\AttributeTok{lower.tail =} \ConstantTok{FALSE}\NormalTok{)}
\end{Highlighting}
\end{Shaded}


\end{document}
